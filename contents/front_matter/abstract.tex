\chapter*{Abstract}\addcontentsline{toc}{chapter}{Abstract}

\marginpar{
	\footnotesize
	\textbf{Keywords:}\\
	Tricuspid Valve Simulation,
	Biomedical Engineering,
	Heart Valve Mechanics,
	Cardiac Simulator,
	Prosthetic Valve Development,
	Material Testing,
	Design Prototyping,
	Physiological Accuracy,

}



{\LARGE This thesis} presents the comprehensive development of a mock tricuspid valve designed to enhance the fidelity and functionality of heart valve simulators used in biomedical engineering education and research. The primary objective of this study was to engineer a tricuspid valve model that authentically replicates the anatomical and biomechanical characteristics of the natural valve while integrating seamlessly into a right heart simulator. This tool is intended for widespread use in educational settings and experimental research, aiming to improve the understanding and handling of cardiac valve dynamics among medical professionals and researchers.

The development process involved a systematic approach combining innovative design methodologies, material testing, and iterative modifications based on empirical data. Challenges such as replicating the intricate motion dynamics of the valve, ensuring material compatibility, and maintaining physiological accuracy under simulated conditions were addressed. Solutions included the utilization of cutting-edge materials that mimic the viscoelastic properties of native valve tissues and the application of advanced modelling techniques to predict and optimize the valve's performance in real-time physiological scenarios.

\newthought{
	Furthermore,} the thesis elaborates on the implications of these developments for future enhancements of cardiac simulators. By achieving a higher degree of realism and functional accuracy, the new model sets a benchmark for developing surgical training aids and pre-surgical planning tools. These advancements are poised to significantly impact cardiac healthcare by improving the precision of surgical interventions. This work contributes to the academic field and has practical implications for improving patient outcomes in cardiac care.