\newcommand{\synopsisDesign}{
    In this chapter the mock valve modelling and prototyping is discussed from conception to completion. Different approaches are disccused in regards to the steps the core design went through;
    \begin{itemize}
        \item Flat Valve: The simplest approach involves creating a flat valve model with leaflets cut to proportion. This design is straightforward to fabricate and provides a basic representation of the \gls{TV} form.
        \item Ballooned Valve: A more anatomically accurate design involves creating a balloon-shaped valve similar to regurgitant valves. This approach aimed to mimic the physiological behavior of the \gls{TV} more closely in the diseased state.
        \item Ballooned Valve with Chordae Tendineae: Building upon the previous design, this approach incorporates chordae tendineae to simulate the \gls{TV}'s complex structure more accurately with practical considerations to how the valve would be modelled, fully represenatively or a simple tri-leaflet valve.
        \item Anatomical Valve without Chordae Tendineae: This approach involved converting \gls{CT} scans of an anatomical valve models, providing a highly realistic representation of the \gls{TV} capturing the likeness very proportionally.
    \end{itemize}
    As well as the design of the other components such as the chordae tendineae and fixturing design.

}
