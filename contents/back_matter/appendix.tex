\chapter{Appendix}\label{ap:appendix}

\vspace{-2.5em}
\newthought{Synopsis} \marginpar{%
	\footnotesize
	\textbf{Resources:}
	\begin{itemize}
		\item \href{https://gitlab.inria.fr/panama-team/blaster}{Code}
		\item \href{https://hal.archives-ouvertes.fr/hal-02469901}{Open-access paper with supplementary material}
	\end{itemize}
	} This appendix briefly describes the Sliding Franck-Wolfe Algorithm used to solve the Non-negativ
The main author of this work is Clement Elvira, to whom I extend my greatest thanks, and I will report it for sake of completeness.




\newthoughtpar{Non-negative Blasso}
To take into account the non-negative constraint on the coefficients, the authors ofhave proposed to slightly modify the SFW algorithm by \textit{i)} removing the absolute value iand 
The reader is referred to for more details.


In particular, using the real part in the implementation allows to remove the imaginary part that may appear due to the imprecision.
