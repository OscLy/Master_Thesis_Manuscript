\chapter{Conclusions and Future Work}\label{ch:conclusion}

In conclusion, this study shows how the design process can be fraught with many more obstacles of many more varieties than expected. It resulted more in learnings of the iterative process of design and prototyping compared to test method development than expected.

The study resulted in a manufacturing process that is consistent and effective in producing anatomically accurate valve models, developed over many different approaches with the strengths taken from each and weaknesses phased out. It also resulted in a test method that was able to effectively test the basic efficacy of the produced valves in a controlled environment and potential to very easily be adapted to introduce a variety tricuspid valve medical devices with simple fixturing and also account for variables like; pressure, flow rate, flow field, and regurgitant characteristics like orifice and volume, once integrated with the right heart simulator.

The implications this type of testing could have on the development of \gls{TTVR} devices is significant. The ability to test the efficacy of these devices in a controlled environment can provide important data that can influence design choices and potentially mitigate against the risk of device failure in the clinical investigations.

\section{Future Research Directions}

Pressure monitoring during test, 36\% glycerin for blood viscosity - rabbah

In vitro micro ct scanning

The next steps in this research would primarily be to fully integrate this valve into the right heart simulator. From there the rig would be enhanced so that fully representative simulations could be conducted where the flow field could be measured with and without \gls{TTVR} devices in place to characterize their efficacy on regurgitant valves. The main prospective device to integrate is the CroiValve DUO device which partially occludes the annulus of the valve reducing the regurgitant flow.

At the moment the CroiValve team don't have the means to conduct work of this nature so the data collected in this study could be used to inform the design inputs of later interations.