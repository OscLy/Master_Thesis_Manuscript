\chapter{Experimental Setup and Testing}\label{ch:testing}
% \openepigraph{Everything in its right place \\
% There are two colours in my head}{Radiohead \textit{Everything In Its Right Place}}
\vspace{-2.5em}
% \newthought{Synopsis}\synopsisMethod

\mynewline
After completing the literature review I decided to go down two main paths: developing a representative valve and a anatomical valve. The representative valve would be a simple model that could be used to test the simulator and the anatomical valve would be a more complex model that could be used to test the simulator and also be used for educational purposes.

\section{Fixturing Assembly}

\section{Rig Assembly}

\section{Challenges and Mitigations}

\begin{itemize}
    \item Leakage: The pulsatile flow loop is a closed system, so leakage has to be carefully monitored. With the nature of the proof of concept rig containing many adapters and connections, it is important to ensure that all connections are secure and that there are no leaks. This was managed with the liberal use of teflon tape and tight fit Buna O-rings. Where there were still leaks, small containers was used to catch the fluid and pour back into the circuit.
    \item Air bubbles: Air bubbles greatly obstructed the coaptation of the valve leaflets. Interupting the fluid flow and also obscuring the view of the valve for evaluation of the coaptation. With the leaks in the system allowing air to enter repeated purging of the system was required to remove air bubbles with a syringe and tube entering through the proximal end of the system.
    \item Chordae Tendineae Adhesion: The adhesion of the most recently attached mock tendineae to the \gls{PU} leaflets had not fully set before the testing began. This resulted in the one of the tendineae detaching during coaptation. As the valve was still functioning the testing was not stopped to mitigate this issue.
\end{itemize}



\section{Biocompatibility}